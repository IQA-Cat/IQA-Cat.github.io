\documentclass[11pt,a4paper,titlepage,openany]{book}
\usepackage[space]{ctex}
\usepackage{mathrsfs,amssymb,amsfonts,amsmath,bm,ntheorem,graphicx}
\usepackage[paperwidth=185mm,paperheight=260mm,text={148mm,210mm},left=21mm,includehead,vmarginratio=1:1]{geometry}
\usepackage{fancyhdr,titlesec,enumerate}
\begin{document}
\pagestyle{fancy}
\fancyhf{}
\fancyhead[EL,OR]{\thepage}
\fancyhead[OC]{\nouppercase{\heiti\rightmark}}
\fancyhead[EC]{\nouppercase{\heiti\leftmark}}
\fancypagestyle{plain}{\renewcommand{\headrulewidth}{0pt}\fancyhf{}}
\theoremstyle{plain}
\newcounter{Proposition}[section]
\newenvironment{Proposition}[1][]{{\par\normalfont\bfseries ����~\stepcounter{Proposition}\arabic{Proposition}#1~~}\kaishu}{\par}
\newcounter{Corollary}[section]
\newenvironment{Corollary}[1][]{{\par\normalfont\bfseries ����~\stepcounter{Corollary}\arabic{Corollary}#1~~}\kaishu}{\par}
\newcounter{Theorem}[section]
\newenvironment{Theorem}[1][]{{\par\normalfont\bfseries ����~\stepcounter{Theorem}\arabic{Theorem}#1~~}\kaishu}{\par}
\newcounter{Lemma}[section]
\newenvironment{Lemma}[1][]{{\par\normalfont\bfseries ����~\stepcounter{Lemma}\arabic{Lemma}#1~~}\kaishu}{\par}
\newcounter{Property}[section]
\newenvironment{Property}[1][]{{\par\normalfont\bfseries ����~\stepcounter{Property}\arabic{Property}#1~~}\kaishu}{\par}
\newcounter{Assertion}[section]
\newenvironment{Assertion}[1][]{{\par\normalfont\bfseries ����~\stepcounter{Assertion}\arabic{Assertion}#1~~}\kaishu}{\par}
\newenvironment{Proof}{{\par{\heiti ֤��}~~}}{\hfill $\square$ \par\hfill\par}
\newcounter{Example}[section]
\newenvironment{Example}[1][]{{\par\normalfont\bfseries ��~\stepcounter{Example}\arabic{Example}#1~~}\songti}{\hfill\par\hfill\par}
\newcounter{Def}[section]
\newenvironment{Def}[1][]{{\par\normalfont\bfseries ����~\stepcounter{Def}\arabic{Def}#1~~\songti}}{\par}
\newcounter{Note}[section]
\newenvironment{Note}[1][]{{\par\normalfont\bfseries ע~\stepcounter{Note}\arabic{Note}#1~~}\songti}{\par}
\title{{\zihao{0}\heiti Homework10}}
\author{��ʥ��}
\maketitle
\chapter{Method Analysis}
\section{Generating}
First we generating the particle in the disc, which we have used the QFT method in last problem. But there is some different, we have to set four directions' particle in the same point, so we have to make the number-set be a four digits that every digit represent a direction. The technique here is the modular that we can get arbitrary digit.\par
It's easy to write the normal particle scattering without the boundary. But there is also many easy way to deal it.\par
We can analysis the topology of the square boundary first, and find there are eight situations, bumping to four lines and four corners, both the same in circle because if we choose large enough number particles, the boundary of  circle behaves like the square. So we just need to discuss the eight situations clearly.\par
We also find a very helpful technique that the particle's momentum at position cross the boundary is just like the mirror symmetry at the position above the boundary.\par
Then for every 100 steps, we can generate a picture to describe the involution of the "dirty" gas.
\chapter{Result}
See Anime.swf
\chapter{Others}
\section{appendix}
anime.swf\par
data\par
hpp.nb\par
HPP.cpp\par
CP12HPP.exe\par
\section{reference}
http://blog.csdn.net/walkandthink/article/details/41847961\par
C++ primer,2003.11,Stanley B.Lippman;Josee Lajoie;Barbara E.Moo,Pearson Education Asia LTD.
\end{document} 